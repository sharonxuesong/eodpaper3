% LaTeX-template for ExoPlanetNews abstract. v1.1 (March 2013)
% Please see the comments below. All lines beginning "%" are not processed or printed
%
%------------------------------------------------------------------------------------------------------------
%
% Remove the "%" from the following 3 lines to preview; please comment out these lines again prior to submission
%\documentclass{article}
%\usepackage{ep}
%\begin{document}
%
%------------------------------------------------------------------------------------------------------------
%
% Title of your paper and short author list (surnames only, no initials)
% Use: 'Author1 et al.' if more than three authors
\title{The Exoplanet Orbit Database \textsc{II}: Updates to
  exoplanets.org}{Han et al.}
%
% or for CONFERENCES use the style:
% \title{Conference Title}{Physical location of Conference}
%
% or for JOB ADVERTS use the style:
% \title{Post Title}{Institute at which post to be held}
%
% or for MISCELLANEOUS ANNOUNCEMENTS use the style:
% \title{Announcement Title}{Institute/Location to which announcement applies}
%
%------------------------------------------------------------------------------------------------------------
%
% Author(s) of your paper (initials then surnames)
% Use \inst{1} etc. for numbering of different institutes.
%{A. Author\inst{1,2}, B. Author\inst{1,3}, C. Author\inst{2} }
\author{Eunkyu Han\inst{1,2}, Sharon X.~Wang\inst{1,2,3}, Jason T.~Wright\inst{1,2,3}, Y.\
  Katherian Feng\inst{1,2}, Ming Zhao\inst{1,2}, Onsi Fakhouri\inst{4}, Jacob I.\ Brown\inst{1,2}, Colin Hancock\inst{1,2}}
%
% or for CONFERENCES use the style:
% \author(Conference Chair(s)}
%
% or for JOB ADVERTS use the style:
% \author{Person offering the job}
%
% or for MISCELLANEOUS ANNOUNCEMENTS use the style:
% \author{Person making the announcement}
%
% Institutes. In the first argument
% Give the numbers as used in the \author command above
%
\instlist{1}{Department of Astronomy \& Astrophysics, 525 Davey
  Lab, The Pennsylvania State University, University Park, PA 16802, USA}
\instlist{2}{Center for Exoplanets and Habitable Worlds, The
  Pennsylvania State University, University Park, PA 16802, USA}
\instlist{3}{Penn State Astrobiology Research Center, The
  Pennsylvania State University, University Park, PA 16802, USA}
\instlist{4}{Pivotal Labs, Cloud Foundry, 875 Howard St., Fourth
  Floor, San Francisco, CA 94103}  
%
% Alternately, if all authors are from the SAME single institute, use:
%\institute{My University, Down the Road, Any Town, UK}
%
%------------------------------------------------------------------------------------------------------------
%
% Status of your paper. 
%First Argument: The journal where it will appear.
% Second Argument. The state - should be one of: "in press" or "published".
%
% If it is already published, please provide us with the ADS-Bibcode,
% otherwise give the arXiv preprint code in the style: "arXiv:nnnn.mmmm"
%
\status{PASP}{in press, arXiv:1409.7709}
%
% or for CONFERENCES use the style:
% \status{Location of Conference}{Dates of Conference}
%
% or for JOB ADVERTS use the style:
% \status{Location of Post}{Job start date}
%
% or for MISCELLANEOUS ANNOUNCEMENTS use the style:
% \status{Location of Announcement}{Relevant date for announcement}
%
% You should use the following abbreviations:
% \mnras   Monthly Notices of the Royal Astronomical Society
% \aj      Astronomical Journal
% \apj     Astrophysical Journal
% \apjl    Astrophysical Journal Letters
% \aa      Astronomy \& Astrophysics
% \aal     Astronomy \& Astrophysics Letters
% \pasp    Publications of the Astronomical Society of the Pacific
% \aas     American Astronomical Society Meeting
% \pasj    Publications of the Astronomical Society of Japan
%
%
%-----------------------------------------------------------------------------------------------------------
%
% The abstract body
%
\abstract{
The Exoplanet Orbit Database (EOD) compiles orbital, transit, host
star, and other parameters of robustly detected exoplanets reported in
the peer-reviewed literature. The EOD can be navigated through the
Exoplanet Data Explorer (EDE) Plotter and Table, available on the
World Wide Web at exoplanets.org. The EOD contains data for 1492
confirmed exoplanets as of July 2014.  The EOD descends from a table
in Butler et al.\ 2002 and the Catalog of Nearby Exoplanets
Butler et al.\ 2006, and the first complete documentation for the EOD
and the EDE was presented in Wright et al.\ 2011. In this work, we
describe our work since then.  We have expanded the scope of the EOD
to include secondary eclipse parameters, asymmetric uncertainties, and
expanded the EDE to include the sample of over 3000 \textit{Kepler} Objects
of Interest (KOIs), and other real planets without good orbital
parameters (such as many of those detected by microlensing and
imaging). Users can download the latest version of the entire EOD as a
single comma separated value file from the front page of
exoplanets.org.
}
%
% Optional Figure caption: (just paste in here - it will be typeset
% separately)
% Figure caption for han2014.eps
An example plot that demonstrates the current secondary eclipse
measurements. Each planet has a set of circles that stand for its
available secondary eclipse measurements in the database,
including \textit{Spitzer} 8.0, 5.8, 4.5, \& 3.6 $\mu$m, and ground-based
$J$, $H$, and $K_{\rm s}$ bands. \textit{Kepler} secondary eclipse
measurements are labeled as solid black dots. A few representative
planets are also annotated (a feature not offered by the EDE
Plotter, but trivially implemented by other image or presentation
software). The planet name will appear when the user points the
cursor at the corresonding data point in the interactive plotting
tools.
%
%------------------------------------------------------------------------------------------------------------
%
% Download: Website with the preprint and/or additional information/data (CHANGE this from the current value)
%
\download{http://exoplanets.org}
%
% Contact: E-Mail of the responsible person (CHANGE this from the current value)
%
\contact{jtwright@astro.psu.edu}
%
%------------------------------------------------------------------------------------------------------------
%
% Remove the "%" from the next line to preview; comment it out again before submission
%
% \end{document}
%
%------------------------------------------------------------------------------------------------------------
