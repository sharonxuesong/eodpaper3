% EOD Documentation Paper #3
% Project initiated Dec 22 2013

%\documentclass{emulateapj}
% Requirement of PASP to use preprint and aastex
\documentclass[11pt,preprint]{aastex}
\usepackage{amsmath,amssymb}
\usepackage{xspace}
\usepackage{graphicx}
\bibliographystyle{apj} 
\usepackage{epstopdf}  
\usepackage{graphicx}
\usepackage{epsfig}
\usepackage{verbatim}
\usepackage{morefloats} % somehow need this to have lots of figures
                        % and tables?
% for attractive links:
\usepackage[colorlinks,urlcolor=blue,citecolor=black,linkcolor=blue]{hyperref}
%\usepackage[nolists]{endfloat} % put floats at end - doesn't work
\interfootnotelinepenalty=10000 % Do not want cross-page footnote!

\usepackage{CJK} 

% definition of symbols
\def\beq{\begin{equation}}
\def\eeq{\end{equation}}
\def\ergcms{erg~cm$^{-2}$~s$^{-1}$}
\def\ergs{erg~s$^{-1}$}
\def\mps{m~s$^{-1}$}
\def\msun{M_{\odot}}
\def\lsun{L_{\odot}}
\def\etal{~et~al.~}
\def\degree{^{\circ}}
\def\leq{\leqslant}
\def\geq{\geqslant}
\def\micron{$\mu$m}
\def\percm2{{\rm cm^{-2}}}
\def\peryr{yr$^{-1}$}
\def\kepler{\textit{Kepler}}
\def\spitzer{\textit{Spitzer}}
\def\micron{$\mu$m}
\def\mjup{$M_{\rm Jup}$}


\maxdeadcycles=1000 % need this to have bunch of figures on each page
                    % at end

\slugcomment{}
\shorttitle{}
\shortauthors{}

%%%%%%%%%%%%%%%%%%%%%%%%%%%%%%%%%%%%%%%%%%%%%%%%%%%%%%%%%%%%%%%%%%%%%%%%%%%%%%%%%%%%%%%%%%%%%%%%%%%%
\begin{document}

\begin{CJK*}{UTF8}{gbsn}

\title{Updates on the Exoplanet Orbit Database}

% leading authors
\author{Eunkyu Han\altaffilmark{1,2}}
\author{Sharon X. Wang\altaffilmark{1,2,3}}
\author{Jason T. Wright\altaffilmark{1,2,3}}
\author{Ying Feng\altaffilmark{1,2}}
\author{Ming Zhao\altaffilmark{1,2}}
\author{Onsi Fakhouri}
\author{and Friends}

\altaffiltext{1}{Department of Astronomy \& Astrophysics, 525 Davey
  Lab, The Pennsylvania State University, University Park, PA 16802, USA;
  Send correspondence to datamaster@exoplanets.org.}
\altaffiltext{2}{Center for Exoplanets and Habitable Worlds, The
  Pennsylvania State University, University Park, PA 16802, USA}
\altaffiltext{3}{Astrobiology Research Center, The
  Pennsylvania State University, University Park, PA 16802, USA} 


%%%%%%%%%%%%%%%%%%%%%%%%%%%%%%%%%%%%%%%%%%%%%%%%%%%%%%%%%%%%%%%%%%%%%%%%%%%%%%%%%%%%%%%%%%%%%%%%%%%%
\begin{abstract}

\end{abstract}  

%%%%%%%%%%%%%%%%%%%%%%%%%%%%%%%%%%%%%%%%%%%%%%%%%%%%%%%%%%%%%%%%%%%%%%%%%%%%%%%%%%%%%%%%%%%%%%%%%%%%
\section{Introduction}\label{sec:intro}

% Recent development in exoplanet discoveries
Since the first peer-reviewed list of exoplanets with robust orbits
\citep{Butler2002,Butler2006}, the count of exoplanets has increased
from less than 200 to over 1000 as of March 2014
(exoplanets.org). Before the launch of NASA's \kepler\ mission
\citep{Borucki2010}, most confirmed exoplanets were discovered via the
precise radial velocity (RV) method. However, since 2009, \kepler\ has
contributed over $\sim$ 400 confirmed exoplanets (ZZZ need citation
update once the $>300$ \kepler\ release comes out) as well as over
3000 exoplanet candidates (\kepler\ Objects of Interest, KOIs;
e.g.~\citealt{Batalha2013}).

% The business of tracking planet discoveries
As the number of exoplanet discoveries keeps rising, it is important
to keep tracking these discoveries and cataloging the orbital
parameters and host star properties of exoplanet systems. There are
several entities that are devoted to this effort, including the
Exoplanet Orbit Database and Exoplanets Data Explorer\footnote{See
  \url{http://exoplanets.org}.} \citep{Wright2011}, the Extrasolar
Planets Encyclopaedia\footnote{See \url{http://exoplanet.eu}.}
\citep{Schneider2011}, the NASA Exoplanet Archive\footnote{See
  \url{http://exoplanetarchive.ipac.caltech.edu}.}
\citep{Akeson2013}, and so on.

% A brief history of EOD


According to Google Analytics, $47,000$ people visited exoplanets.org
in 2013.

What this paper is about: an update on EOD and the services our
website, exoplanets.org, provide.




%%%%%%%%%%%%%%%%%%%%%%%%%%%%%%%%%%%%%%%%%%%%%%%%%%%%%%%%%%%%%%%%%%%%%%%%%%%%%%%%%%%%%%%%%%%%%%%%%%%%
\section{Updates on the EOD}\label{sec:update}

Scope of EOD has changed in the following ways:

1. Definition of planets: not strictly smaller than 24 \mjup\ any
more? Or still true? We still require smaller than 24 \mjup.

2. Host stars: not only `normal stars' any more? Not any more -
neutron stars are there (at least)! This requirement is completely gone.

3. Change on definition of `well-determined' orbital parameters? No. Still
qualitative, still the same as described in Section~2 of \cite{Wright2011}.

4. Inclusion of planets discovered beyond RV and transit.

Purposes of EOD have not changed?

We only describe new additions and revised fields here, but list all
fields. For details on other fields, see \cite{Wright2011}. We divide
the fields into categories in consistency with the interactive table
on exoplanets.org. Some fields are not explicitly included in the
interactive table (e.g., the URLs), but are explicitly listed in the
downloadable {\tt .csv} file.\footnote{The entire dataset provided by
  exoplanets.org is available for download in a single CSV file at
  \url{http://exoplanets.org/csv-files/exoplanets.csv}} Make a note
here that though we focus on EOD in this Section, the fields apply to
all planets on exoplanets.org.

%------------------------------------------------------------------------------
\subsection{Discovery and References}

We report general information of when, by whom and how the planets were
discovered and provide the references. We describe the added fields since \cite{Wright2011} below. \\

{\bf KEPID} stands for KEPler ID which is the integer number assigned for a Kepler host star. KEPID is only available for the unconfirmed KOI which we import directly from the Exoplanet Archive. Therefore, none of the EOD planets and the confirmed Kepler planets has KEPID.


{\bf KOI} stands for Kepler Objects of Interest which is a star with transit signals detected by \kepler\ but have not been confirmed to due to exoplanets. The KOI field contains a floating point number that consist of a whole number designated to the host star followed by a decimal number denoting a candidate planet. For example, KOI 102 is a star suspected to host two exoplanets and KOI 102.01 is the inner planet candidate and KOI 102.02 is the outer one. Once a candidate is confirmed, the NAME field of the planet is replaced by the official \kepler\ ID ('Kepler' followed by a hyphen, an integer and a letter). The KOI number is used as OTHERNAME. None of the KOIs are in the EOD and for more information see Section~\ref{sec:kepler}. 

\begin{comment}
\textit {Sharon: you need to say what KOI stores. When writing each
  field, keep in mind you need to cover the following so that the
  reader will completely understand: (1) what the field stands for
  (e.g., in case of an abbreviation like KOI); (2) what the field
  physically means, like `KOI' means planet candidates that have transit
  signals as detected by \kepler\ but have not been confirmed, and
  make sure to be thorough at what you say -- sometimes things are not
  that obvious to an outsider reader, e.g., think if a cosmologist
  would understand what you're saying when reading it; (3) what the
  field actually contains, e.g. KOI contains the KOI number designated
  by the \kepler\ team (right?), it is a floating point number, and
  sometimes maybe even worth giving an example, e.g. KOI 30.1 or
  something...; (4) why we added this new field, if this information
  is helpful, or if it's a revised field since \cite{Wright2011}, what
  is changed. Also, use \kepler\ (which is a defined symbol in our ms.tex file),
  instead of typing out Kepler, since you need italic.
  Don't be afraid to write more! Being concise is important, but it's
  always easier to trim down things than to add things on, especially
  for a documentation type of paper like this. Let's be thorough at
  first, then try to reach optimal conciseness.}
 \end{comment}

KDE: obselete field! Detele from the table.

{\bf EOD} is a flag which indicates whether an exoplanet has well-characterized orbital parameters (e.g. period (PER)). EOD equals 1 means the exoplanet is in the EOD. We added this field because exoplanets.org now also includes robustly discovered planets that do not have well-characterized orbits, and also \kepler\ planet candidates (see Section~\ref{sec:kepler} for more details).


{\bf MICROLENSING, IMAGING, TIMING and ASTROMETRY} are flags which indicate that a planet detected via theses methods. For example, MICROLENSING $=$ 1 means the planet is detected by a microlensing event. However, we require the detection to be robust, as determined by the QUAlity Control Board of the Exoplanet Database, QUACBED. 


%------------------------------------------------------------------------------
\subsection{Orbit Parameters}
We report the orbit parameters that determine the shape of the planet's orbit. Depending on the detection methods, different parameters are given by the literatures. For instance, if a planet is detected by the radial velocity method, we get the semi-amplitude, whereas the transit method gives the inclination of the orbit. 

Added fields since \cite{Wright2011}:


MASS is the mass of the planet in the unit of Jupiter mass. Previously, we only reported the minimum mass (MSINI) of the planets. We now report mass as well. If a planet is discovered by transit, the planet mass is given by the literature. However, if a planet is detected by radial velocity
method, the minimum mass is given in which case we calculate the mass
by dividing the minimum mass by the orbit inclination. MASS is set to
MSINI if the orbital inclination is not given and the MASSREF is set
to ``Set to MSINI; I unknown''. For KOIs without MSINI measured, the Exoplanet Archive reports MASS which is derived from the mass-radius relation. 


SEP is the projected separation between the host star and the planet in the unit of AU. For directly imaged planets, the literatures give the projected separation and we report this value. There are cases where only the semi-major axis is given without the projected separation, then SEP is set to the semi-major axis. For the cases where only the projected separation is given such as imaged planets, then the semi-major axis is set to SEP. 

BIGOM is the longitude of the ascending node, which together with other orbital parameters (PER, T0, K, OM, ECC and I) determines the the planet's orbit. 
ZZ why did we start reporting this?
We thought we would get them eventually but we haven't seen any paper that actually reports this.

LAMBDA is the projected spin-orbit misalignment in the unit of degree. We report the projected spin-orbit misalignment of transiting systems. We follow the definition of Fabrycky and Winn 2009. The spin-orbit angle is defined to be the angle between the stellar spin axis and the orbital axis. However, since the spin-orbit angle is not directly measurable, and therefore lambda, the projected spin-orbit angle was introduced. In the observer-oriented frame, Z axis points toward the observer, X axis points along the intersection of the sky plane and the planet's orbital plane, and Y axis completes a right handed-triad. Lambda is measured clockwise on the sky from the Y axis to the projected stellar rotational axis. While maintaining LAMBDA in our database, we figured out that literatures have conflicted definitions of LAMBDA with $\beta$, which is equal to $\lambda$ = -$\beta$. We sorted these cases out and made sure that LABDA is following the definition of Fabrycky and Winn 2009. 


I is the orbital inclination. This field is not newly added but we chose to describe it here since our database now has the inclination for planets detected by methods other than transit. Previously, only transiting systems used to have the inclination measured. However, now sometimes for
non-transiting systems the values of I are also available. For instance, if the system was first detected by radial velocity and has a follow-up transit observation, then the inclination can be measured. To sum up, all the transiting planets have measured orbital inclination and currently, we also have one microlensing planet with measured inclination. *** give the example!

you can only measure inclination with transit and imaging but there is only one microlensing planet with the inclination


%------------------------------------------------------------------------------
\subsection{Transit Parameters}

Added fields since \cite{Wright2011}:

DR

RR is the ratio of the planetary radius and the stellar radius. Previously, we only reported the transit depth, which is the square of the planetary radius and the stellar radius.
=== Why did we add this field? 

%------------------------------------------------------------------------------
\subsection{Secondary Eclipse}
We have added an entire new set of fields for the secondary eclipse. Since \cite{Wright2011}, 

This is a entire new set of fields since \cite{Wright2011}:

SE flag indicates whether the system has detected secondary eclipse in any of the 2MASS J, H, and K$_s$, Kepler photometry band, and 4 Spitzer IRAC bands. SE is a boolean and being true indicates the system has detected secondary eclipse. 

SEDEPTH J H KS KP 36 45 58 80 fields are the secondary eclipse depth measured in the corresponding wavelength band. J, K, and K$_s$ are the depth measured in the 2MASS wavelength band in near infrared each centered at 1.25\micron, 1.65\micron, and 2.15\micron, respectively. KP is the depth measured in the Kepler photometry band which is centered at 

SET is the epoch of secondary eclipse center in HJD. 

%------------------------------------------------------------------------------

\subsection{Stellar Properties}
ZZZ Stellar parameters and stellar properties have been merged. 

Stellar parameters section is consisted of 3 parts: Stellar properties, Stellar magnitudes, and Coordinates and Catalog. This section is to provide our users detailed information about the exoplanet hosting stars. Stellar properties section contains name, mass, and consistent set of metalicities (FE) , effective temperature (TEFF), surface gravity (LOGG), projected rotational velocity (VSINI), and the systematic radial velocity (GAMMA). Stellar magnitudes section includes optical magnitudes of the star (V and BMV) as well as 2MASS infrared (J, H, and K) and Kepler bandpass magnitude (KP). Coordinates and Catalog section includes the right ascension (RA) and the declination (DEC) of the host star, parallax (PAR) and distance (DIST) and the integers designated by different catalogs. 

Added fields since \cite{Wright2011}:


GAMMA is the newly added field under Stellar properties. It is the radial velocity of the center of mass of the planetary system and is reported in km/s.  


Difference from Stellar Parameters: these are derived parameters based
on spectroscopy and stellar models. 
==>> Not always true for RSTAR because some literatures give the value. 
Added fields since \cite{Wright2011}:

RSTAR is the radius of the star in the solar radius. 

RHOSTAR is the density of the star in the unit of $g/cm^3$

%------------------------------------------------------------------------------
\subsection {Stellar Magnitudes}

We provide the brightness of the host star measured in different wavelengths. We report optical magnitudes (V and B-V), 2MASS J,H, and K magnitudes centered at 1.25\micron, 1.65\micron, and 2.15\micron, respectively. 

Added field: 
KP is the stellar magnitude measured through the Kepler bandpass which covers 4000 angstroms to 8500 angstroms. 

%------------------------------------------------------------------------------
\subsection{Coordinates and Catalogs}

Added fields since \cite{Wright2011}:

DIST is the distance of the host star in the unit of pc. It sometimes is calculated from the parallax that come from the Hipparcos data set by van Leeuwen (2009).


%------------------------------------------------------------------------------
\subsection{Uncertainties}

We now report uncertainties for a certain field X as XUPPER, XLOWER
and UX. XUPPER and XLOWER store the upper and lower 1$\sigma$
uncertainties, and they are different when the error bars are
asymmetric. UX contains the 1$\sigma$ uncertainty, and it
equals to $({\rm XUPPER+XLOWER})/2$ in the case of asymmetric error
bars. Fields that have XUPPER, XLOWER and UX are: (list of fields).

%------------------------------------------------------------------------------
\subsection{References}

Added field: MONTH, PLANETDISCMETH, STARDISCMETH.

The fields XREF and XURL store the peer-reviewed reference and link to
the publication for the field X. The following fields now have XREF
and XURL: (previously in \citealt{Wright2011}) DIST, TRANSIT, BINARY;
(new) fields in stellar parameters: V, VSINI, GAMMA; all orbit pars except for
trend and freeze\_ecc; all transit parameters; all SE parameters;
fields for stellar properties: MSTAR, RSTAR, TEFF, RHOSTAR, LOGG, FE.

VREF is not the same as REF for J H KS, which do not have XREF
fields. See \cite{Wright2011} for references of J H KS mags -- I think
he mentioned it there.

Other general reference fields: In \cite{Wright2011}: FIRSTREF,
FIRSTURL, ORBREF, ORBURL. New: SPECREF, SPECURL contain source of most
of the spectroscopic parameters listed for convenience; actual
references in the REF fields take precedence over this one.

We now report the month when the peer-reviewed paper is accepted to
ApJ to better track of the trend in exoplanetary detections. Moreover,
we report the discovery methods for the host star and the planet
separately: STARDISCMETH and PLANETDISCMETH. See Section~ for more
details.  


%------------------------------------------------------------------------------
\subsection{Revised and Removed Fields}

These include:

OTHERNAME - This is now listed as other name of planet instead of star;
** OTHTERNAME has always been the name of the star I think...?

DISCMETH - We have replaced DISCMETH with PLANETDISCMETH and STARDISCMETH to better report how the host star and the planet are detected. 

S - changed to SHK; ZZZ What's the source for SHK and RHK? Still discovery
paper and Butler 2006? ZZZ SHK is a poorly maintained field, and it is
unreferenced at the moment though we plan to include propoer reference
in the future.

SPSTAR - stellar type of star (unvetted), gone;

NSTEDID - replaced by EANAME (Exoplanet Archive).

MASSREF, MASSURL - originally for stellar mass reference, now for planet mass.

%------------------------------------------------------------------------------
Figure: Figure of single planet page, to show updated transit
parameters and RV curves etc.?


%%%%%%%%%%%%%%%%%%%%%%%%%%%%%%%%%%%%%%%%%%%%%%%%%%%%%%%%%%%%%%%%%%%%%%%%%%%%%%%%%%%%%%%%%%%%%%%%%%%%
\section{Inclusion of the Kepler Planet Candidates and Other Exoplanets on Exoplanets.org}\label{sec:kepler}

Since \cite{Wright2011}, as more planets were discovered
by different methods other than radial velocity and transit, we have expanded this information to better ingest those planets into our database.

Planets primarily detected by these methods do not have robust orbital parameters, they are categorized as ``Other Planets'' on exoplanets.org. 

If planets have follow up radial velocity or transit observations and
thus have good orbital parameters, they have EOD flags. See
Section~\ref{sec:kepler} for more information about `Other Planets'. 


We now include \kepler\ candidates... They can be displayed by
checking the `Kepler' option for the interactive data table.

We also include robustly detected planets discovered microlensing,
imaging, timing, astrometric motion. They can be displayed by check
the `Other' option for the table. They are not included in the EOD
(especially some microlensing and imaging planets) because of poorly
characterized orbits...

Also add our criteria for vetting IMAGING planets, which can be found
here: http://exoplanets.org/methodology.html; do we have any similar
criteria for Microlensing, timing and astrometry planets? Like mass
cut etc.?

Figure: some comparison figures of Kepler candidate properties
vs. EOD planet properties, such as mass-period, mass-radius?


%%%%%%%%%%%%%%%%%%%%%%%%%%%%%%%%%%%%%%%%%%%%%%%%%%%%%%%%%%%%%%%%%%%%%%%%%%%%%%%%%%%%%%%%%%%%%%%%%%%%
\section{Updates on the Website}

Onsi? Essential updates on the website.


%%%%%%%%%%%%%%%%%%%%%%%%%%%%%%%%%%%%%%%%%%%%%%%%%%%%%%%%%%%%%%%%%%%%%%%%%%%%%%%%%%%%%%%%%%%%%%%%%%%%
\section{Conclusion and Future}\label{sec:conclusion}

Any planned future improvements and updates?


%%%%%%%%%%%%%%%%%%%%%%%%%%%%%%%%%%%%%%%%%%%%%%%%%%%%%%%%%%%%%%%%%%%%%%%%%%%%%%%%%%%%%%%%%%%%%%%%%%%%
\acknowledgments

NSF funding?

SXW acknowledges support from PSARC.


%%%%%%%%%%%%%%%%%%%%%%%%%%%%%%%%%%%%%%%%%%%%%%%%%%%%%%%%%%%%%%%%%%%%%%%%%%%%%%%%%%%%%%%%%%%%%%%%%%%%
% References:
%\newpage
% List references here
\begin{thebibliography}

\bibitem[Akeson et al.(2013)]{Akeson2013} Akeson, R.~L., et al.\ 2013, 
\pasp, 125, 989 % NASA Exoplanet Archive

\bibitem[Batalha et al.(2013)]{Batalha2013} Batalha, N.~M., et al.\ 2013, 
\apjs, 204, 24 % example of KOI release

\bibitem[Bennett et al.(2010)]{Bennett2010} Bennett, D.~P., et al.\ 2010, 
\apj, 713, 837 % OGLE microlensing planet with inclination measured

\bibitem[Borucki et al.(2010)]{Borucki2010} Borucki, W.~J., et al.\ 2010, 
Science, 327, 977 % Kepler instroduction and first results

\bibitem[Butler et al.(2002)]{Butler2002} Butler, R.~P., et al.\ 2002, 
\apj, 578, 565 % first peer-reviewed list of planets with robust orbits

\bibitem[Butler et al.(2006)]{Butler2006} Butler, R.~P., et al.\ 2006, 
\apj, 646, 505 % EOD first paper

\bibitem[Chauvin et al.(2012)]{Chauvin2012} Chauvin, G., et al.\ 2012, 
\aap, 542, A41 % beta Pic orbital pars

\bibitem[Collier Cameron et al.(2010)]{Collier Cameron2010} Collier 
Cameron, A., et al.\ 2010, \mnras, 407, 507 % WASP-33b having no RV
                                % detection and K upper limit

\bibitem[Cosentino et al.(2012)]{Cosentino2012} Cosentino, R., et al.\ 
2012, \procspie, 8446, 1V % HARPS-North

\bibitem[Dawson \& Murray-Clay(2013)]{Dawson2013} Dawson, R.~I., \&
  Murray-Clay, R.~A.\ 2013, \apjl, 767, L24  % use of EOD

\bibitem[Fabrycky \& Winn(2009)]{Fabrycky2009} Fabrycky, D.~C., \&
  Winn, J.~N.\ 2009, \apj, 696, 1230 % spin-orbit misalignment lambda definition
  
\bibitem[Green et al.(2012)]{Green2012} Green, J., et al.\ 2012, 
arXiv:1208.4012   % WFIRST

\bibitem[Horner et al.(2012)]{Horner2012} Horner, J., Wittenmyer, R.~A., 
Hinse, T.~C., \& Tinney, C.~G.\ 2012, \mnras, 425, 749 % dynamic
                                % analysis on timing planet

\bibitem[Howard(2013)]{Howard2013} Howard, A.~W.\ 2013, Science, 340,
  572 % use of EOD

\bibitem[Johnson et al.(2011)]{Johnson2011} Johnson, J.~A., et al.\ 2011, 
\apjs, 197, 26 % HD 116029B having ECC upper limit (circular orbit)
  
\bibitem[Kane(2014)]{Kane2014} Kane, S.~R.\ 2014, \apj, 782, 111 % use
                                % of EOD
  
\bibitem[Kipping(2013)]{Kipping2013} Kipping, D.~M.\ 2013, \mnras,
  434, L51 % use of EOD

\bibitem[Lagrange et al.(2009)]{Lagrange2009} Lagrange, A.-M., Gratadour, D., Chauvin, G., et al.\ 2009, \aap, 493, L21 %beta Pic b, imaging planet with I measured

\bibitem[Lissauer et al.(2011)]{Lissauer2011} Lissauer, J.~J., et al.\ 
2011, \apjs, 197, 8 % Mass-Radius for 1 Re < R < 6 Re

\bibitem[Macintosh et al.(2014)]{Macintosh2014} Macintosh, B., Gemini 
Planet Imager instrument team, Planet Imager Exoplanet Survey, G., 
\& Observatory, G.\ 2014, American Astronomical Society Meeting
Abstracts, 223, \#229.02  % GPI

\bibitem[Marcy et al.(2014)]{Marcy2014} Marcy, G.~W., et al.\ 2014, \apjs, 
210, 20 % 42 RV confirmed small Kepler planets

\bibitem[Mahadevan et al.(2012)]{Mahadevan2012} Mahadevan, S., 
Ramsey, L., Bender, C., et al.\ 2012, \procspie, 8446, %HPF  

\bibitem[Mordasini et al.(2012)]{Mordasini2012} Mordasini, C., Alibert, Y., 
Georgy, C., Dittkrist, K.-M., Klahr, H., \& Henning, T.\ 2012, \aap, 547, A112
% Mass-Radius for 6 Re < R < 11.684 Re

\bibitem[Moutou et al.(2011)]{Moutou2011} Moutou, C., D{\'{\i}}az, R.~F., Udry, S., et al.\ 2011, \aap, 533, A113  %Lambda example

\bibitem[Nielsen et al.(2014)]{Nielsen2014} Nielsen, E.~L., et al.\ 2014, 
arXiv:1403.7195 % beta Pic orbit pars
  
\bibitem[Perryman \& ESA(1997)]{Perryman1997} Perryman, M.~A.~C., \&
  ESA 1997, ESA Special Publication, 1200 % Hipparcos B-V reference

\bibitem[Ricker(2014)]{Ricker2014} Ricker, G.~R.\ 2014, Journal of the 
American Association of Variable Star Observers (JAAVSO), 42, 234 % TESS

\bibitem[Rowe et al.(2014)]{Rowe2014} Rowe, J.~F., et al.\ 2014, 
arXiv:1402.6534 % 715 new Kepler planets (multis)

\bibitem[Sanchis-Ojeda et al.(2012)]{Sanchis-Ojeda2012} Sanchis-Ojeda, R., 
et al.\ 2012, \nat, 487, 449 % Kepler 30 system lambda measurement
                             % with star spot transit

\bibitem[Schneider et al.(2011)]{Schneider2011} Schneider, J., Dedieu, C., 
Le Sidaner, P., Savalle, R., \& Zolotukhin, I.\ 2011, \aap, 532, A79
% exoplanet.eu

\bibitem[Skrutskie et al.(2006)]{Skrutskie2006} Skrutskie, M.~F., et al.\ 
2006, \aj, 131, 1163 % 2MASS

\bibitem[Sotin et al.(2007)]{Sotin2007} Sotin, C., Grasset, O., 
\& Mocquet, A.\ 2007, \icarus, 191, 337 % Mass-Radius for M<1Me, R<1Re

\bibitem[van Leeuwen(2009)]{van Leeuwen2009} van Leeuwen, F.\ 2009, \aap, 
497, 209 % Hipparcos catalog

\bibitem[Weiss \& Marcy(2014)]{Weiss2014} Weiss, L.~M., \& Marcy,
  G.~W.\ 2014, \apjl, 783, L6  % Mass-Radius for R < 4Re to be implemented

\bibitem[Winn et al.(2005)]{Winn2005} Winn, J.~N., et al.\ 2005, \apj, 631, 
1215 % HD 209458b, the first Rossiter-McLaughlin lambda measurement

\bibitem[Wittenmyer et al.(2012)]{Wittenmyer2012} Wittenmyer, R.~A., 
Horner, J., Marshall, J.~P., Butters, O.~W., 
\& Tinney, C.~G.\ 2012, \mnras, 419, 3258 % dynamic analysis on timing
                                % planets

\bibitem[Wittenmyer et al.(2013)]{Wittenmyer2013} Wittenmyer, R.~A., 
Horner, J., \& Marshall, J.~P.\ 2013, \mnras, 431, 2150 % dynamic
                                % analysis on timing planets

\bibitem[Wright et al.(2011)]{Wright2011} Wright, J.~T., et al.\ 2011,
  \pasp, 123, 412 % EOD 2011 paper

\bibitem[Wright \& Gaudi(2013)]{Wright2013} Wright, J.~T., \& Gaudi, B.~S.\ 2013,
  Planets, Stars and Stellar Systems.~Volume 3: Solar and Stellar
  Planetary Systems, 489 % Review article with definition of orbital elements

\bibitem[Wright et al.(2014)]{Wright2014} Wright, J., et al.\ 2014, 
American Astronomical Society Meeting Abstracts, 223, \#148.31

\end{thebibliography}



%%%%%%%%%%%%%%%%%%%%%%%%%%%%%%%%%%%%%%%%%%%%%%%%%%%%%%%%%%%%%%%%%%%%%%%%%%%%%%%%%%%%%%%%%%%%%%%%%%%%
% Table
\clearpage

\begin{deluxetable}{llll}
\tabletypesize{\scriptsize}
%\rotate
\tablewidth{0pt}
\tablecaption{Fields of Exoplanet Orbit Database\label{tab:par}}
\tablehead{
  \colhead{Field\tablenotemark{a}} & \colhead{Data Type} & \colhead{Meaning} &
  \colhead{Related Fields\tablenotemark{b}}
}
\startdata
%
\sidehead{\textbf{Planet Information}}
NAME\dotfill & String & Name of planet & \nodata \\
OTHERNAME\tablenotemark{c}\dotfill & String & Other commonly used name of planet & \nodata \\
COMP\dotfill & String & Component name of planet ($b$, $c$, etc.) & \nodata \\
HD\dotfill & Long Integer & Henry Draper number of sar & \nodata \\
HR\dotfill & Integer & Bright Star Catalog number of star & \nodata \\
HIPP\dotfill & Long Integer & \textit{Hipparcos} catalog number of
star & \nodata \\
SAO\dotfill & Long Integer & SAO catalog number of star & \nodata \\
GL\dotfill & Float & GJ or Gliese catalog number of star & \nodata \\
BINARY\dotfill & Boolean & Star known ot be binary? & -REF, -URL \\
RA\dotfill & Double & J2000 right ascension in decimal hours & \nodata \\
RA\_STRING\dotfill & String & J2000 right ascension in sexagesimal string & \nodata  \\
DEC\dotfill & Double & J2000 declination in decimal degrees & \nodata \\
DEC\_STRING\dotfill & String & J2000 declination in sexagesimal string & \nodata \\
KEPID\dotfill & Long Integer & The unique \kepler\ star identifier & \nodata \\
KOI\dotfill & Float & KOI object number & \nodata \\
KDE\dotfill & Boolean & If true, planet appears in the \kepler\ archive & \nodata \\
EOD\dotfill & Boolean & If true, planet is included in the EOD & \nodata \\
MICROLENSING\dotfill & Boolean & If true, planet was detected via microlensing & \nodata \\
IMAGING\dotfill & Boolean & If true, planet was detected via imaging & \nodata \\
TIMING\dotfill & Boolean & If true, planet was detected via timing & \nodata \\
ASTROMETRY\dotfill & Boolean & If true, planet was detected via astrometric motion & \nodata \\
%
\sidehead{\textbf{Stellar Parameters}}
BMV\dotfill & Float & $B-V$ color & \nodata \\
V\dotfill & Float & $V$ magnitude & -REF,-URL \\
J\dotfill & Float & $J$ magnitude & \nodata \\
H\dotfill & Float & $H$ magnitude & \nodata \\
KS\dotfill & Float & $K_S$ magnitude & \nodata \\
DIST\dotfill & Float & Distance to host star based on parallax in parsecs & -UPPER, etc. \\
PAR\dotfill & Float & Parallax in mas & -UPPER,-LOWER,U- \\
VSINI\dotfill & Float & Projected equatorial rotational velocity of
star in k\mps & -UPPER, etc. \\
GAMMA\dotfill & Float & Systemic radial velocity in k\mps & -UPPER, etc. \\
%
\sidehead{\textbf{Orbit Parameters}}
PER\dotfill & Double & Orbital period in days & -UPPER, etc. \\
ECC\dotfill & Float & Orbital eccentricity & -UPPER, etc. \\
OM\dotfill & Float & Argument of periastron in degrees & -UPPER, etc. \\
K\dotfill & Float & Semiamplitude of stellar reflex motion in \mps & -UPPER, etc. \\
T0\dotfill & Double & Epoch of periastron in HJD\tablenotemark{d}$-2,440,000$ & -UPPER, etc. \\
DVDT\dotfill & Float & Magnitude of linear trend in \mps\ day$^{-1}$ & -UPPER, etc. \\
I\dotfill & Float & Orbital inclination in degrees & -UPPER, etc. \\
MSINI\dotfill & Float & Minimum mass (as calculated from the mass
function) in \mjup & -UPPER, etc. \\
A\dotfill & Float & Orbital semimajor axis in AU & -UPPER, etc. \\
MASS\dotfill & Float & Mass of planet in \mjup & -UPPER, etc. \\
SEP\dotfill & Float & Separation between host star and planet in AU & -UPPER, etc. \\
TREND\dotfill  & Boolean & Linear trend in fit? & \nodata \\
LAMBDA\dotfill & Float & Projected spin-orbit misalignment in degrees & -UPPER, etc. \\
BIGOM\dotfill & Float & Longitude of ascending node in degrees & -UPPER, etc. \\
FREEZE\_ECC\dotfill & Boolean & Eccentricity frozen in fit? & \nodata \\
%
\sidehead{\textbf{Transit Parameters}}
DEPTH\dotfill & Float & Transit depth, $(R_p/R_*)^2$ & -UPPER, etc. \\
T14\dotfill & Float & Time of transit from first to fourth contact in days & -UPPER, etc. \\
TT\dotfill & Float & Epoch of transit center in
HJD\tablenotemark{d}$-2,440,000$ & -UPPER, etc. \\
R\dotfill & Float & Radius of planet in Jupiter radii & -UPPER, etc. \\
AR\dotfill & Float & $(a/R_*)$ & -UPPER, etc. \\
B\dotfill & Float & Impact parameter of transit & -UPPER, etc. \\
DENSITY\dotfill & Float & Density of planet in g cc$^{-1}$ &
-UPPER, etc. \\
GRAVITY\dotfill & Float & $\log{g}$ (surface gravity) of planet in cgs unit &
-UPPER, etc. \\
TRANSIT\dotfill & Boolean & Is the planet known to transit? & -REF,-URL \\
DR\dotfill & Float & Distance during transit in stellar radii & -UPPER, etc. \\
RR\dotfill & Float & $(R_p/R_*)$ & -UPPER, etc. \\
%
\sidehead{\textbf{Secondary Eclipse}}
SE\dotfill & Boolean & If true, at least one secondary eclipse has
been detected & -REF,-URL \\
SEDEPTHJ\dotfill & Float & Secondary eclipse depth in $J$ band & -UPPER, etc. \\
SEDEPTHH\dotfill & Float & Secondary eclipse depth in $H$ band & -UPPER, etc. \\
SEDEPTHKS\dotfill & Float & Secondary eclipse depth in $K_S$
band & -UPPER, etc. \\
SEDEPTHKP\dotfill & Float & Secondary eclipse depth in the
\kepler\ photometry band & -UPPER, etc. \\
SEDEPTH36\dotfill & Float & Secondary eclipse depth in
\spitzer\ IRAC1 3.6 \micron\ band & -UPPER, etc. \\
SEDEPTH45\dotfill & Float & Secondary eclipse depth in
\spitzer\ IRAC2 4.5 \micron\ band & -UPPER, etc. \\
SEDEPTH58\dotfill & Float & Secondary eclipse depth in
\spitzer\ IRAC3 5.8 \micron\ band & -UPPER, etc. \\
SEDEPTH80\dotfill & Float & Secondary eclipse depth in
\spitzer\ IRAC4 8.0 \micron\ band & -UPPER, etc. \\
SET\dotfill & Double & Epoch of secondary eclipse center in
HJD\tablenotemark{d}$-2,440,000$ & -UPPER, etc. \\
%
\sidehead{\textbf{Fit \& References}}
RMS\dotfill & Float & Root-mean-square residuals to orbital RV fit & \nodata \\
CHI2\dotfill & Float & $\chi_{\nu}^2$ to orbital RV fit & \nodata \\
NOBS\dotfill & Integer & Number of observations used in fit & \nodata \\
NCOMP\dotfill & Integer & Number of planetary companions known & \nodata \\
MULT\dotfill & Boolean & Multiple planets in system? & \nodata \\
PLANETDISCMETH\dotfill & String & Method of discovery of planet & \nodata \\
STARDISCMETH\dotfill & String & Method of discovery of first planet in system & \nodata \\
DATE\dotfill & Integer & Year of publication of FIRSTREF & \nodata \\
MONTH\dotfill & Integer & Month of publication of FIRSTREF & \nodata \\
FIRSTREF\dotfill & String & First peer-reviewed publication of
planetary orbit & FIRSTURL \\
ORBREF\dotfill & String & Peer-reviewed origin of orbital parameters & ORBURL \\
%
\sidehead{\textbf{Stellar Properties}}
STAR\dotfill & String & Standard name for host star & \nodata \\
MSTAR\dotfill & Float & Mass of host star in solar mass & -UPPER, etc. \\
RSTAR\dotfill & Float & Radius of host star in solar radii & -UPPER, etc. \\
TEFF\dotfill & Float & Effective temperature of host star & -UPPER, etc. \\
RHOSTAR\dotfill & Float & Density of host star & -UPPER, etc. \\
LOGG\dotfill & Float & Spectroscopic $\log{g}$ (surface gravity) of
host star in cgs unit & -UPPER, etc. \\
FE\dotfill & Float & Iron abundance (or metallicity) of star & -UPPER, etc. \\
SHK\dotfill & Float & Mount Wilson Ca \sc{II} $S$-value & \nodata \\
RHK\dotfill & Float & Chromospheric activity of star as $R'_{HK}$ & \nodata \\
KP\dotfill & Float & \kepler\ bandpass magnitude & \nodata \\
SPECREF\dotfill & String & Source of most of the spectroscopic parameters & SPECURL \\
%
\sidehead{\textbf{Links}}
JSNAME\dotfill & String & Name of host star used in the Extrasolar
Planets Encyclopaedia & EPEURL \\
ETDNAME\dotfill & String & Name of planet used in the Exoplanet
Transit Database & ETDURL \\
SIMBADNAME\dotfill & String & Valid SIMBAD name of host star (or
planet, if available) & SIMBADURL \\
EANAME\dotfill & String & Name of planet used in the Exoplanet
Archive database & EAURL \\
\enddata
\tablenotetext{a}{Fields in bold are updates or new additions from
  \cite{Wright2011}. See Section~\ref{sec:update} for more details.}
\tablenotetext{b}{-UPPER etc. means ``-UPPER, -LOWER, U-, -REF, -URL",
where ``-" stands for the name of the field listed in the first
column. They are...}
\tablenotetext{c}{Originally as other commonly used name for star in \cite{Wright2011};
  changed to other commonly used name for planet.}
\tablenotetext{d}{The same as in \cite{Wright2011}, HJD/BJD etc.\ are
  still not consistent.}
\end{deluxetable}



%%%%%%%%%%%%%%%%%%%%%%%%%%%%%%%%%%%%%%%%%%%%%%%%%%%%%%%%%%%%%%%%%%%%%%%%%%%%%%%%%%%%%%%%%%%%%%%%%%%%
\end{CJK*}

\end{document}
